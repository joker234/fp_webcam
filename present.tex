\documentclass{beamer}
\usepackage{fontspec, siunitx, luacode, xcolor, marvosym, graphicx, verbatim}
\setbeamertemplate{navigation symbols}{}
\sisetup{output-decimal-marker={,}}

\usetheme{Luebeck} %dafür muss 2mal kompiliert werden!
\useoutertheme{split} %zugegeben, das verändert nichts, aber alle anderen Optionen sind voll hässlich und wir mögen das Lübeck-Theme =/ Außerdem ist die Beamer Dokumentation auf Spanisch oO
\useinnertheme{rounded}

\title{Lego NXT - Zählwerk}
\author{Philip Bell, Saskia Klaus, Johannes Visintini}

\begin{document}
\begin{frame}
\titlepage
\end{frame}

\begin{frame}
\tableofcontents
\end{frame}

\section{Aufgabenstellung und Werdegang}
\begin{frame}{Aufgabenstellung}
\begin{itemize}
	\item Bau eines Zählwerks nach dem Vorbild des Time Twister
	\item Hochzählen bei Auslösung eines Sensors
\end{itemize}
\includegraphics[scale=0.3]{0937.jpg}\\
{\tiny\copyright \hspace{.3em} Hans Andersson 2011}
\end{frame}

\begin{frame}{Werdegang - Milestones}
\begin{itemize}
	\item Mitte Dezember: Beginn des Praktikums
	\item Mitte Januar: Ende der Einarbeitungsphase
	\item Anfang Februar: Fertigstellung des Grundgerüsts des Roboters
	\item Mitte Februar: Funktionsweise der Sensoren verstanden haben und ausnutzen können, währenddessen programmieren
	\item Mitte März: Roboter zu Ende programmieren
	\item Ende März: Präsentation des Roboters und Dokumentation erstellen
\end{itemize}
\end{frame}

\section{Bau des Zählwerks}
\begin{frame}{Bau des Zählwerks}
\begin{itemize}
	\item Bau der Säulen nach Anleitung
	\item Abweichung beim Rahmen wegen geringerer Anzahl an Säulen
	\item jede Säule besteht aus fünf Segmenten
	\item jedes Segment besteht aus acht grauen bzw. schwarzen Blöcken\\
	\includegraphics[scale=0.2]{segpres.jpg}\\
	\item<2> Zusammensetzen der Segmente, sodass die Säulen sich in gewünschter Weise drehen können
	\item<2> Säulen und Brick am Rahmen befestigen
\end{itemize}
\end{frame}

\begin{frame}{Bau des Zählwerks}
Probleme:
\begin{itemize}
	\item Instabilität
	\item Anbau der Sensoren
\end{itemize}
\end{frame}

\begin{frame}{Bau des Zählwerks}
\begin{itemize}
	\item Stabilisierungsmaßnahmen:
	\begin{itemize}
		\item Weitere Reihe von horizontalen und vertikalen Streben
		\item NXT-Block als Gegengewicht
		\item hölzerne Verkleidung
	\end{itemize}
	\item<2> Sensorenanbringungen:
	\begin{itemize}
		\item Mikro: Gute Erreichbarkeit, Abstand zu den Motoren
		\item Ultraschallsensor: Gute Erreichbarkeit
		\item Farbsensor: Gute Erreichbarkeit, keine direkte Lichteinstrahlung\\ 
		\Rightarrow gemeinsame, externe Anbringung an der Holzverkleidung
		\item Tastsensor: Flexibel durch Kabel als einzige Verbindung zum Zählwerk
	\end{itemize}
\end{itemize}
\end{frame}


\section{Algorithmus}
\begin{frame}{Algorithmus}
\begin{itemize}
	\item Säulenmechanik:
	\begin{itemize}
		\item Indirekte Steuerung
		\item Positionskompatibilität\\
		\includegraphics[scale=.4]{seg.jpg}
	\end{itemize}
\end{itemize}
\end{frame}
\begin{frame}[fragile]{Algorithmus}
\begin{itemize}
	\item Zustandskodierung:
	\begin{itemize}
		\item Fünfstellige Kombination aus Zahlen von Null bis Drei
		\item Bestimmen der Zahlen für jedes Segment einzeln
		\item Ermöglicht Kodierung von nicht-Zahlen\\
		\includegraphics[scale=.5]{zustand.jpg}
	\end{itemize}
	\item<2> Signumsberechnung:
	\begin{itemize}
		\item Bestimmung der Anfangsdrehrichtung
		\item Fehleranfälligster Teil des Algorithmus
		\vspace{.3em}
		\item {\tiny\begin{verbatim}
				for(i=4;i>=0;i--) {
				    int a = val1[i]-val2[i];
				    if(a==1 || a==-3) { // rechts
				        return -1;
				    } else if(a==-1 || a==3) { // links
				        return 1;
				    }
				}
			\end{verbatim}
			}
	\end{itemize}
\end{itemize}
\end{frame}
\begin{frame}[fragile]{Algorithmus}
\begin{itemize}
	\item Übergangskodierung:
	\begin{itemize}
		\item Maximal sechsstellige Zahlenkombination
		\item Signum nimmt erste Stelle ein
		\item Weitere Stellen werden mit Anzahlen notwendiger Vierteldrehungen gefüllt
		\item Berechnung der Vierteldrehung durch Simulation des Überganges
		\vspace{1em}
		\item {\tiny\begin{verbatim}
			void up(int* aktuell, int signum, int pos) {
			    aktuell[pos] = mod4(aktuell[pos] + signum);
			    int x;
			    if (pos<4 && abs(aktuell[pos]-aktuell[pos+1])==2) { // <=4 oder <4
			        up(aktuell, signum, ++pos);
			    }
			}
		\end{verbatim}
		}
		\vspace{1em}
		\item Übergabe der Übergangskodierungen an den NXT-Block als Array 
	\end{itemize}
\end{itemize}
\end{frame}


\section{Programm}
\begin{frame}[fragile]{Programm}
\begin{itemize}
	\item Bewegungen
	\begin{itemize}
		\item Problematik: Parallelausführung führt zu unterschiedlichen Befehlen für den gleichen Motor
		\item Umsetzung durch tasks, in verbindung mit Mutexen
		\vspace{.5em}
		\item {\tiny\begin{verbatim}
			task jumpA() {
			    Acquire(MotA);
			    int ziel = a;
			    int i;
			    int sgn = uebergang[zif_a*16*6+ziel*6];
			    for(i=1;i<6;i++) {
			        RotateMotor(OUT_A,85,(-1)*sgn*uebergang[zif_a*16*6+ziel*6+i]*270);
			        sgn *= -1;
			    }
			    zif_a = ziel;
			    Release(MotA);
			}
			
		\end{verbatim}
		}
	\end{itemize}
	\item<2> Initialisierung
	\begin{itemize}
		\item Problematik
		\item Umsetzung
	\end{itemize}
\end{itemize}
\end{frame}
\begin{frame}[fragile]{Programm}
\begin{itemize}
	\item Modi
	\begin{itemize}
		\item Idee: nur Hochzählen ist langweilig, mehrere Funktionalitäten
		\item Umsetzung: Farbkarten, die den Modus des Programms wechseln\\
		     Problematik: Farben wurden nur unzureichend erkannnt.\\
   			  → Lösung: Minimierung von Fremdlicht
		\item Modi: Hochzählen (Gelb), Zahlenrotation (Grün), Mathemodus -,+ (Rot)
		\item Zusatz-Modus: Bluetooth (Blau)\\
		     dieser bietet durch eine beliebige Stringeingabe viele Möglichkeiten,\\
			 z.B. die Implementierung eines Hexadezimalmodus
	\end{itemize}
\end{itemize}
\end{frame}
\begin{frame}{Programm}
\begin{itemize}
	\item Bluetooth-Modus: \\
	\vspace{.5em}
	\includegraphics[scale=.3]{NXT_Mailbox.jpg}\\
	{\tiny\copyright \hspace{.3em} Ferdinand Stückler \& Google}
\end{itemize}
\end{frame}

\section{Vorführung \& Abschluss}
\begin{frame}{Vorführung}
	Live-Vorführung unseres Praktikums ...
\end{frame}
\begin{frame}{Bildverzeichnis}
Bild-Quellen:
\begin{itemize}
	\item Tilted Twister (Folie 3): \copyright \hspace{.3em} Hans Andersson 2011\\
		{\tiny http://tiltedtwister.com/timetwister.html (Stand: 23.04.2013)}
	\item NXT Mailbox (Folie 17): \copyright \hspace{.3em} Ferdinand Stückler \& Google\\
		{\tiny https://play.google.com/store/apps/details?id=NXT.BTMailbox (Stand: 23.04.2013)}
	\item sonstige Bilder: \copyright \hspace{.3em} Johannes Kamera
\end{itemize}
\end{frame}
\begin{frame}{Ende}
	\begin{center}\huge Ende\end{center}
\end{frame}



%Inhaltsverzeichnis:
%Aufgabe
%Bau des Zählwerks
%Algorithmus
% Programm
%Vorführung

%\begin{verbatim}
%\usepackage{siunitx, amsmath}
%\sisetup{output-decimal-marker={,}}
%$\SI{5,32}{\electronvolt}$
%\end{verbatim}

%\begin{frame}[fragile]
%\begin{luacode}
%for k=350,780,5 do
%tex.print("\\color[wave]{",k,"}\\Celtcross Ende ")
%end
%\end{luacode}
%\end{frame}
\end{document}
